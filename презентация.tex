\documentclass{beamer}
\usepackage[english,russian]{babel}
\usepackage[utf8]{inputenc}
% Стиль презентации
\usetheme{PaloAlto}
\begin{document}
\title{Билет №3}  
\author{Комаров Александр}
\institute{СПбГЭТУ «ЛЭТИ»}
\date{Санкт-Петербург, 2019} 
\frame{\titlepage} 
\begin{frame}{Содержание}
 \begin{thebibliography}{10}
\beamertemplatebookbibitems
\bibitem{1}
{\sc 1}, {\em Понятие кодирования информации.Основные задачи кодирования: экономичность представления и обработки информации, защита от ошибок}.
\bibitem{2}
{\sc 2}, {\em Персональное и коллективное общение в сети.Электронная почта: назначение, основные функции, режим работы.Организация доступа. Технология работы с электронной почтой.}.
\end{thebibliography}
\end{frame}

\begin{frame}{Понятие кодирования информации}
Существует огромное количество разнообразных данных (текст, звук, изображения и др.)- Работа с разнородными данными очень трудоемка. Для автоматизации работы необходимо унифицировать их форму представления — для этого используется прием кодирования.
Кодирование — это выражение данных одного типа через данные другого типа (например, с помощью алфавита кодируются звуки какого-либо языка, а музыкальные произведения кодируются нотами). Закодированные данные передаются в виде сигналов. В компьютере используется двоичная система кодирования, потому что она наиболее точно передает работу электронно-магнитных устройств, которые могут находиться в двух состояниях: пассивном (нет сигнала) и активном (есть сигнал). 
\end{frame}

\begin{frame}{Понятие кодирования информации.}
Любая информация всегда хранится в виде кодов. В виде кодов хранятся и изображения.
Компьютеры работают с цифровой информацией, а не с аналоговой.Для перевода используют дискретизацию во времени (в случае аналогового сообщения), квантование по значению (уровню) и последующее кодирование.\\
1.Дискретизация – замена аналогового сигнала совокупностью его отсчётов.\\
2.Квантование – округление дискретного сигнала до ближайшего разрешенного уровня.\\
3.Кодирование – замена квантованных значений последовательностью кодовых групп
\end{frame}

\begin{frame}{Задачи кодирования}
Основной задачейкодирования является оптимальное использование информационных характеристик источников сообщений и каналов связи для построения кодов, обеспечивающих заданную достоверность передаваемой информации с максимально возможной скоростью и минимально возможной стоимостью передачи сообщений.
Кодированный по определённому закону сигнал приобретает свойства обнаружения, а иногда и исправления ошибок, возникающих в процессе передачи и приёма сообщений.
Применение специальных кодов, известных только соответствующим абонентам, обеспечивает секретность передачи. Кроме того, кодирование сигналов может также решать задачу согласования параметров канала связи и сигналов: применив тот или иной метод кодирования, удаётся их согласовать. 
\end{frame}


\begin{frame}{Общение в сети}
IM-сети в основном предназначены для индивидуального общения, а программы организации чатов, Web-форумов и Web- конференций, гостевых книг, блогов - для общения групп людей, имеющих какие-либо общие интересы.
Чат (chat) - специальной программное средство для организации виртуального общения одновременно большого числа людей в Интернете - существует в двух разных видах:\\
1. IRC - Internet Relay Chat - использует специальные серверы и клиенты для своей работы, подключенные к Интернету;\\
2. Web-чат - реализован, как программное обеспечение, работающее на обычных Web-страницах.\\
У этих двух видов много общего: для участия в разговорах нужно зарегистрироваться и задать себе ник (псевдоним), разговоры ведутся в общих тематических «комнатах» - окнах системы (или каналах в IRC), можно уйти в «приват» - поговорить тет-а-тет в отдельной «комнате» и пр.
\end{frame}

\begin{frame}{Электронная почта}
Электронная почта (Electronic mail, или сокращенно E-mail) - это способ отправки и получения сообщений с помощью компьютерной сети. Каждый абонент, подключенный к почтовому серверу, имеет свой электронный адрес или, образно говоря, почтовый ящик. Доступ к этому ящику защищен паролем абонента. Благодаря электронному адресу можно идентифицировать любого абонента в сети (все адреса должны быть оригинальными). Абонент и сервер для обмена сообщениями используют кабель (если абонент подключен к ЛВС с почтовым сервером) или обычную телефонную линию (в случае удаленного соединения).
\end{frame}


\begin{frame}{Электронная почта}
Каждый абонент, подключенный к почтовому серверу, имеет свой электронный адрес или, образно говоря, почтовый ящик. Доступ к этому ящику защищен паролем абонента. Благодаря электронному адресу можно идентифицировать любого абонента в сети (все адреса должны быть оригинальными). Абонент и сервер для обмена сообщениями используют кабель (если абонент подключен к ЛВС с почтовым сервером) или обычную телефонную линию (в случае удаленного соединения).  
\end{frame}


\begin{frame}{Электронная почта}
К преимуществам электронной почты относятся:\\
1.Высокая скорость пересылки сообщений\\
2.На обмен информацией затрачиваются несколько минут, в то время как при использовании обычной авиапочты на пересылку корреспонденции затрачивается не менее недели\\
3.Экономичность\\
4.Одна страница текста передается за доли секунды, а по телефону необходимо затратить несколько минут\\
5.Возможность работать с текстом, а не со звуком (голосом), что позволяет обдумать и отредактировать ответ во время обмена информацией, особенно на иностранном языке, не требуется 6.Мгновенная реакция\\
7.Одновременная пересылка по нескольким адресам\\
8.Удобный способ для передачи приглашений, рекламных сообщений и т.п.\\
9.Передача файлов вместе с письмом.
\end{frame}
\end{document}